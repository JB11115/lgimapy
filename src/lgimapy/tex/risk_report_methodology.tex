\documentclass[12pt]{article}
\usepackage[left=0.5cm, right=0.5cm, top=0.3cm, bottom=0.2cm, landscape]{geometry}
\usepackage[table]{xcolor}
\usepackage[open]{bookmark}
\usepackage{
    amsmath,
    amsthm,
    amssymb,
    adjustbox,
    array,
    background,
    booktabs,
    bm,
    caption,
    colortbl,
    dsfont,
    enumitem,
    epigraph,
    fancyhdr,
    float,
    graphicx,
    hyperref,
    makecell,
    marvosym,
    MnSymbol,
    nicefrac,
    ragged2e,
    subcaption,
    tabu,
    titlesec,
    transparent,
    wasysym,
    xcolor
}
\PassOptionsToPackage{hyphens}{url}\usepackage{hyperref}

\backgroundsetup{contents={}}

%% Define default caption settings for figures and tables.
\captionsetup{
    justification=raggedright,
    singlelinecheck=false,
    font={footnotesize, bf},
    aboveskip=0pt,
    belowskip=0pt,
    labelformat=empty
}
\captionsetup[subfigure]{
    justification=raggedright,
    singlelinecheck=false,
    font={footnotesize, bf},
    aboveskip=0pt,
    belowskip=0pt,
    labelformat=empty
}
%% Define columns with fixed widths for tables.
\newcolumntype{L}[1]{>{\raggedright\arraybackslash}p{#1}}
\newcolumntype{C}[1]{>{\centering\arraybackslash}p{#1}}
\newcolumntype{R}[1]{>{\raggedleft\arraybackslash}p{#1}}

%% Change separations distances for better presentation.
\setlength{\textfloatsep}{0.1mm}
\setlength{\floatsep}{1mm}
\setlength{\intextsep}{2mm}

%% Declare operators for mathematical expressions.
\newcommand{\N}{\mathbb{N}}
\newcommand{\Z}{\mathbb{Z}}
\DeclareMathOperator*{\argmax}{argmax}
\DeclareMathOperator*{\argmin}{argmin}
\setcounter{MaxMatrixCols}{20}

%% Define custom colors.
\definecolor{lightgray}{gray}{0.9}
\definecolor{steelblue}{HTML}{0C70D5}
\definecolor{firebrick}{HTML}{E85650}
\definecolor{orchid}{HTML}{9A2BE6}
\definecolor{orange}{HTML}{E69A2B}
\definecolor{babyblue}{HTML}{85C7DB}
\definecolor{salmon}{HTML}{DB8585}
\definecolor{eggplant}{HTML}{815a71}
\definecolor{mauve}{HTML}{a78197}
\definecolor{oldmauve}{HTML}{4C243B}
\definecolor{navy}{HTML}{192E5B}
\definecolor{lightpink}{HTML}{FBC2EF}
\definecolor{mintgreen}{HTML}{D4F0B7}
\definecolor{lightblue}{HTML}{72A2C0}
\definecolor{tan}{HTML}{E8D0AC}
\definecolor{opal}{HTML}{9BC1BC}
\definecolor{sage}{HTML}{CACAAA}
\definecolor{oceangreen}{HTML}{59C9A5}
\definecolor{magicmint}{HTML}{A0EEC0}
\definecolor{persiangreen}{HTML}{1B998B}
\definecolor{paleblue}{HTML}{B1DDF1}
\definecolor{champagne}{HTML}{EEE3AB}
\definecolor{powderblue}{HTML}{A9D2D5}

%% Define function for perctile bars in tables.
\newlength{\barw}
\setlength{\barw}{0.15mm}
\def\pctbar#1#2{%%
    {\color{gray}\rule{#2\barw}{#1pt}} #2\%}

%% Define function for divergent bars in tables.
\def\bar#1#2{%
    \color{#2}\rule{#1\barw}{7pt}
}

\def\divbar#1#2{%
    \ifnum#1<0{%
            \bar{\the\numexpr50+#1}{white}
        \bar{-#1}{#2}
        \bar{50}{white}
    }
    \else{%
            \bar{50}{white}
        \bar{#1}{#2}
        \bar{\the\numexpr50-#1}{white}
    }
    \fi
}

\pagenumbering{gobble}

\fancyhf{}
\setlength{\headheight}{2cm}
\fancyhead[L]{Strategy Risk Report}
\fancyhead[R]{EOD February 18, 2022}

            \setlength\footskip{0pt}
            \fancyfoot[R]{
                \raisebox{2cm}[0pt][0pt]
                {\includegraphics[width=0.065\textwidth]
                {"<ROOT>/fig/logos/LG_umbrella"}}
            }

\pagestyle{fancy}






\begin{document}


\section*{Methodology}

\pdfbookmark[0]{Methodology}{99999999}

\begin{table}[H]
\centering
\begin{adjustbox}{width =\textwidth}
\begin{tabu} to \linewidth{c|c|c|c|l}
\toprule

\textbf{Metric} &
\textbf{Notation} &
\textbf{Units} &
\textbf{Equation} &
\textbf{Description}\\

\specialrule{2.5pt}{1pt}{1pt}
DTS (\%) & $DTS_{pct}$ & \% & $\dfrac{DTS_{Port}}{DTS_{BM}}$ &
    \makecell[l]{
    Measure of Portfolio's Beta proxied by the Portfolio's DTS as a
    percentage of the Benchmark's DTS.\\ Generally 97\% is considered neutral for
    small ($<10$ bp) moves, while 100\% is neutral for large selloffs/rallies.
    }\\
\midrule
DTS OW (abs) & $DTS^{OW}_{abs}$ & bp $\cdot$ yr & $DTS_{Port} - DTS_{BM}$ &
    Portfolio's DTS less the Benchmark's DTS.\\
\midrule
DTS OW (dur) & $ DTS^{OW}_{dur}$ & yr & $\dfrac{DTS^{OW}_{abs}}{OAS_{BM}}$ &
    \makecell[l]{
        DTS overweight expressed in Duration terms. For example, a -0.5 value
        can be interpreted such that for every\\100 bp move wider in BM spread,
        you would expect to make +50 bp of performance. This measure is most \\
        usefull for large (10+ bp) moves in the BM.
    }\\

\specialrule{2.5pt}{1pt}{1pt}
Barbell (\%) & $Barb$ & \% (in DTS terms) &
    \makecell{
        $DTS_{pct} - DTS_{implied}$
        \vspace{0.2cm} \\
        \scriptsize
        where $DTS_{implied} = 1 - \dfrac{OAD^{OW}_{Tsy}}{OAD_{BM}}$
    } &
    \makecell[l]{
    Using our treasury position (in duration terms), we calculate what the DTS
    of the portfolio would be if we owned\\ the BM pro rata after accounting for
    treasuries. We then calculate how much the portfolio's DTS exceeds this \\
    implied ``neutral" level from our given treasury position.
    }\\
\midrule
Tracking Error & $TE_{n}$ & bp (annualized) &
    \makecell{
        $ \sqrt{252} * \sigma (\vec{R}_{Port} - \vec{R}_{BM})$
        \vspace{0.2cm} \\
        \scriptsize
        where $\vec{R}$ is daily total returns \\
        \scriptsize
        over an $n$ month window
    } &
    \makecell[l]{
        Annualized tracking error computed from daily performance numbers. This
        is a backwards looking measure\\ which proxies the Alpha generated in
        our portfolios. A $n=3$ month window is used to reduce volatilty in \\
        the metric at the cost of slower updating to changes in the portfolio.
    }\\
\midrule
Normalized TE & $\hat{TE_{n}}$ & \% & $\dfrac{TE_{n=1}}{BM_{OAS}}$ &
    \makecell[l]{
        Tracking error normalized by BM spread level to correct for changing volatilty
        dynamics over time.\\ A 1 month TE was empirically chosen as it allows
        TE to update quicker to match spread level of the BM.
    }\\

\specialrule{2.5pt}{1pt}{1pt}
Carry & $Carry$ & bp & $OAS_{Port} - OAS_{BM}$ & Spread carry over the BM.\\

\specialrule{2.5pt}{1pt}{1pt}
Curve Duration & $CurveDur_{n}$ & yr & it's complicated &
    \makecell[l]{
        A duration measure for curve positioning around a pivot point $n$. For
        example, a Curve Duration (10yr) value of +0.10 \\can be interpreted such
        that if the treasury curve steepened 100 bp around the 10 yr, you would
        expect to make +10 bp of\\performance. Positive (negative) values indicate
        the portfolio is in a steepener (flattener) around the pivot point.
    }\\

\specialrule{2.5pt}{1pt}{1pt}
Performance & $Perf(t)$ & bp &
    \makecell{
        $R^{Port}_{t} -  R^{BM}_{t}$
        \vspace{0.2cm} \\
        \scriptsize
        where $R^{Port}_{t} =
            \dfrac{\sum^{Port}_{b} R_{t}(b) \cdot MV^{Port}_{t}(b)}{\sum^{Port}_{b} MV^{Port}_{t}(b)}$ \\
        \scriptsize
        and $R^{BM}_{t} =
            \dfrac{\sum^{BM}_{b} R_{t}(b) \cdot MV^{BM}_{t}(b)}{\sum^{BM}_{b} MV^{BM}_{t}(b)}$ \\
        \scriptsize
        and $b$ refers to individual CUSIPs
    } &
    \makecell[l]{
    Daily performance for day $t$ approximated by market value weighting
    the total returns of each bond (including treasuries)\\ held in the
    portfolio vs the benchmark. Intra-day transactions are not accounted for.
    }\\
\midrule
Daily Attribution & $Attr(t)$ & bp &
    \makecell{
        $\sum^{Bonds}_{b} \Delta OAS_{t}(b) \cdot \overline{OAD}^{OW}_{t}(b)$
        \vspace{0.4cm} \\
        \scriptsize
        where $\Delta OAS_{t}(b) = OAS_{t}(b) - OAS_{t-1}(b)$
        \vspace{0.2cm} \\
        \scriptsize
        and $\overline{OAD}^{OW}_{t}(b) =
            \dfrac{OAD^{OW}_{t}(b) + OAD^{OW}_{t-1}(b)}{2}$ \\
    } &
    \makecell[l]{
        Daily attribution for day $t$ approximated by multiplying the spread
        change (from closing spreads) by our duration \\ overweight in individual
        CUSIPs, and summing over the specified basket (ticker, sector, etc.).
        This back-of-the-envelope \\ spread methodology does not take into account
        intra-day moves or transactions levels.
    }\\
\midrule
Attribution & $Attr(t-n, t)$ & bp & $\sum^{n}_{i=0} Attr(t-i)$ &
    \makecell[l]{
        Attribution over a specified period, calculated by simply summing
        daily attribution values.
    }\\

\bottomrule
\end{tabu}
\end{adjustbox}
\end{table}

\end{document}
